% !TEX TS-program = pdflatex
% !TEX encoding = UTF-8 Unicode

% This is a simple template for a LaTeX document using the "article" class.
% See "book", "report", "letter" for other types of document.

\documentclass[11pt]{article} % use larger type; default would be 10pt

\usepackage[utf8]{inputenc} % set input encoding (not needed with XeLaTeX)

%%% Examples of Article customizations
% These packages are optional, depending whether you want the features they provide.
% See the LaTeX Companion or other references for full information.

%%% PAGE DIMENSIONS
\usepackage{geometry} % to change the page dimensions
\geometry{a4paper} % or letterpaper (US) or a5paper or....
% \geometry{margin=2in} % for example, change the margins to 2 inches all round
% \geometry{landscape} % set up the page for landscape
%   read geometry.pdf for detailed page layout information

\usepackage{graphicx} % support the \includegraphics command and options

% \usepackage[parfill]{parskip} % Activate to begin paragraphs with an empty line rather than an indent

%%% PACKAGES
\usepackage{booktabs} % for much better looking tables
\usepackage{array} % for better arrays (eg matrices) in maths
\usepackage{paralist} % very flexible & customisable lists (eg. enumerate/itemize, etc.)
\usepackage{verbatim} % adds environment for commenting out blocks of text & for better verbatim
\usepackage{subfig} % make it possible to include more than one captioned figure/table in a single float
% These packages are all incorporated in the memoir class to one degree or another...

\usepackage{color}

%%% HEADERS & FOOTERS
\usepackage{fancyhdr} % This should be set AFTER setting up the page geometry
\pagestyle{fancy} % options: empty , plain , fancy
\renewcommand{\headrulewidth}{0pt} % customise the layout...
\lhead{}\chead{}\rhead{}
\lfoot{}\cfoot{\thepage}\rfoot{}

%%% SECTION TITLE APPEARANCE
\usepackage{sectsty}
\allsectionsfont{\sffamily\mdseries\upshape} % (See the fntguide.pdf for font help)
% (This matches ConTeXt defaults)

%%% ToC (table of contents) APPEARANCE
\usepackage[nottoc,notlof,notlot]{tocbibind} % Put the bibliography in the ToC
\usepackage[titles,subfigure]{tocloft} % Alter the style of the Table of Contents
\renewcommand{\cftsecfont}{\rmfamily\mdseries\upshape}
\renewcommand{\cftsecpagefont}{\rmfamily\mdseries\upshape} % No bold!

%%% END Article customizations

%%% The "real" document content comes below...

\title{Lab Four\\
Computational Probability and Statistics \\
CIS 2033, Section 002}
\author{Due: 9:00 AM, Friday, October 31, 2014}
\date{} % Activate to display a given date or no date (if empty),
         % otherwise the current date is printed 

\begin{document}
\maketitle

\paragraph*{Question 1}
Let $X$ be a discrete random variable. For each of the following cases when
\begin{itemize}
\item $X$ is a Binomial distribution, $X \sim Bin(n, p), n = 100, p = 0.5$;
\item $X$ is a Geometric distribution, $X \sim Geo(p), p = 0.5$,
\end{itemize}
you have to compute the TRUE and EMPIRICAL values for the mean and variance of $X$. For the true values, you can calculate them manually or you can first generate such a distribution with the specific parameters and then use the functions provided by the MATLAB to compute its mean and variance. For the empirical values, you first randomly generate $N$ samples from such a distribution, and then use the {\it mean} and {\it var} functions to compute the empirical mean and variance on those samples, respectively. You have to repeat this process for 10 times and obtain the average value of the computed empirical mean and variance over the 10 repeats. Please experimented with $N = [5,10,50,100,500,1000,5000]$ and then plot a 2D figure, where x-axis denotes N and the y-axis denotes the empirical values for mean or variance. Finally, you have to add a line of the true values for the mean or variance. Please use different colors for the true and empirical values. 
For each case, you have to submit 
\begin{enumerate}
\item MATLAB codes, which should be put in script files (.m); 
\item Two figures, which should be in eps format (.eps). One is for the empirical and true mean values and the other is for the empirical and true variance values. 
\end{enumerate}

\paragraph*{Question 2}
Let $X$ be a continuous random variable. For each of the following cases when
\begin{itemize}
%\item $X$ is a Uniform distribution, $X \sim U(\alpha, \beta), \alpha=0, \beta = 1$;
\item $X$ is an Exponential distribution, $X \sim Exp(\lambda), \lambda = 2$;
%\item $X$ is a Pareto distribution, $X \sim Par(\alpha), \alpha = 2$;
\item $X$ is a Normal distribution, $X \sim N(\mu, \sigma^2), \mu=0, \sigma = 1$,
\end{itemize}
you have to compute the TRUE and EMPIRICAL values for the mean and variance. For the true values, you can manually compute them or you can first generate such a distribution with the specific parameters and then use the functions provided by the MATLAB to compute the true value of its mean and variance. For the empirical values, you first randomly generate $N$ samples from such a distribution, and then use the {\it mean} and {\it var} functions to compute the empirical mean and variance on those samples, respectively. You have to repeat this process for 10 times and obtain the average value of the computed empirical mean and variance over the 10 repeats. Please experimented with $N = [5,10,50,100,500,1000,5000]$ and then plot a 2D figure, where x-axis denotes N and the y-axis denotes the empirical values for mean or variance. Then, you have to add a line of the true values for the mean or variance. Please use different colors for the true and empirical values. 
For each case, you have to submit 
\begin{enumerate}
\item MATLAB codes, which should be put in script files (.m); 
\item Two figures, which should be in eps format (.eps). One is for the empirical and true mean values and the other is for the empirical and true variance values. 
\end{enumerate}



\end{document}
