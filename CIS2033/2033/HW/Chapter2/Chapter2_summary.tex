\documentclass{article} % For LaTeX2e
\usepackage{nips12submit_e,times}
%\documentstyle[nips12submit_09,times,art10]{article} % For LaTeX 2.09

\usepackage{amssymb}
\usepackage{amsmath}


\title{Chapter 2\\Outcomes, Events and Probability}


\author{
%Department of Computer and Information Science\\
%Temple University\\
%Philadelphia, PA 19122 \\
%\texttt{minxiao@temple.edu} 
}

% The \author macro works with any number of authors. There are two commands
% used to separate the names and addresses of multiple authors: \And and \AND.
%
% Using \And between authors leaves it to \LaTeX{} to determine where to break
% the lines. Using \AND forces a linebreak at that point. So, if \LaTeX{}
% puts 3 of 4 authors names on the first line, and the last on the second
% line, try using \AND instead of \And before the third author name.

\newcommand{\fix}{\marginpar{FIX}}
\newcommand{\new}{\marginpar{NEW}}

\nipsfinalcopy % Uncomment for camera-ready version

\begin{document}


\maketitle

\section{Summary}

\begin{itemize}
\item
\begin{itemize}
\item A {\em sample space} ($\Omega$) is a set of all the outcomes for a certain experiment.  
\item A {\em event} is a subset of the sample space. 
\end{itemize}
\item 
\begin{itemize}
\item The {\em intersection} of event $A$ and event $B$ ($A \cap B$) is a set containing all the outcomes when both $A$ and $B$ occur. 
\item The {\em union} of event $A$ and $B$  ($A \cup B$) is a set containing all the outcomes when $A$ or $B$ occurs. 
\item The {\em complement} of event $A$, $A^c = \{\omega \in \Omega: \omega \notin A\}$, is a set containing all the outcomes when $A$ does not occur.
\end{itemize}
\item 
\begin{itemize}
\item The event $A$ and event $B$ are {\em disjoint} or {\em mutually exclusive} if $A \cap B = \emptyset$.
\item The event $A$ implies the event $B$ if $A \subset B$. 
\end{itemize}
\item {\em DeMorgan's Laws}\\
For any two events $A$ and $B$, we have 
\begin{itemize}
\item $(A \cup B)^c = A^c \cap B^c$ and 
\item $(A \cap B)^c = A^c \cup B^c$. 
\end{itemize}
\item 
\begin{itemize}
\item A {\em probability function} P on a finite sample space $\Omega$ assigns to each event $A$ in $\Omega$ a number $P(A)$ in [0,1] such that
\begin{itemize}
\item $P(\Omega) = 1$, and
\item $P(A\cup B) = P(A) + P(B)$ if A and B are disjoint. 
\end{itemize}
The number P(A) is called the probability that A occurs. 
\item A probability function on an infinite (or finite) sample space $\Omega$ assings to each event $A$ in $\Omega$ a number $P(A)$ in [0,1] such that
\begin{itemize}
\item $P(\Omega)$ = 1, and
\item $P\left(A_1 \cup A_2 \cup A_3 \cup \cdots \right) = P(A_1) + P(A_2) + P(A_3) + \cdots$ if $A_1, A_2, A_3, \ldots$ are disjoint events. 
\end{itemize}
\end{itemize}
\item
\begin{itemize}
\item $P(A) = P(A \cap B) + P(A \cap B^c)$.
\item $P(A \cup B) = P(B) + P(A \cap B^c)$. 
\item $P(A^c) = 1 - P(A)$.
\end{itemize}
\item The probability of union.\\
For any two events A and B we have \\$P(A \cup B) = P(A) + P(B) - P(A \cap B)$. \\
For any three events A and B and C we have \\$P(A \cup B \cup C) = P(A) + P(B) + P(C) - P(A \cap B) - P(A \cap C) - P(B \cap C) + P(A \cap B \cap C)$.
\item Products of sample spaces\\
When we perform an experiment $n$ times, then the correponding sample space is $\Omega = \Omega_1 \times \Omega_2 \times \cdots \Omega_n$, where $\Omega_i$ for $i=1, \ldots, n$ is a copy of the sample space of the original experiment. Moreover, the probability of the outcomes $(\omega_1, \omega_2, \ldots, \omega_n)$ is $P\left((\omega_1, \omega_2, \ldots, \omega_n)\right) = p_1 \cdot p_2 \cdots p_n$ if each $\omega_i$ has probability $p_i$.  
\end{itemize}

\end{document}
