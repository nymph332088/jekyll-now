\documentclass{article} % For LaTeX2e
\usepackage{nips12submit_e,times}
\usepackage{graphicx}
\usepackage{amsmath}
\usepackage{amssymb}
\usepackage{multirow}
%\documentstyle[nips12submit_09,times,art10]{article} % For LaTeX 2.09


\title{Notes and Sample Questions \\
Chapter 10}


\author{}

% The \author macro works with any number of authors. There are two commands
% used to separate the names and addresses of multiple authors: \And and \AND.
%
% Using \And between authors leaves it to \LaTeX{} to determine where to break
% the lines. Using \AND forces a linebreak at that point. So, if \LaTeX{}
% puts 3 of 4 authors names on the first line, and the last on the second
% line, try using \AND instead of \And before the third author name.

\newcommand{\fix}{\marginpar{FIX}}
\newcommand{\new}{\marginpar{NEW}}

\nipsfinalcopy % Uncomment for camera-ready version

\begin{document}

\maketitle

\section{Useful equations for Chapter 10}
If we are given the joint distribution of two random variables X,Y, we are able to calculate the expectation of any function $g(X, Y)$ w.r.t $X,Y$:\\
\begin{tabular}{|l|}
\hline
\textbf{Definition:} $E[g(X,Y)] =\begin{cases} \sum_x\sum_yg(x,y)P(X=x,Y=y), & X~ \text{is discrete}\\ \int\int_xy g(x,y)f(x,y)dx, &X~ \text{is continuous} \end{cases}$\\
\textbf{Definition:} $Var[g(X,Y)]=E[(g(X,Y)-E[g(X,Y)])^2] = E[g(X,Y)^2] - E[g(X,Y)]^2$\\
\textbf{Definition:} $Cov(X,Y) =E[(X-E[X])(Y-E[Y])] = E[XY]-E[X]E[Y], \rho(X,Y) = \frac{Cov(X,Y)}{\sqrt{Var[X]Var[Y]}}$\\
\textbf{Definition:} $Cov(g(X),g(Y)) = E[g(X)g(Y)] - E[g(X)]E[g(Y)]$\\
\textbf{Lemma:} $Cov(rX+s,tY+u) = rtCov(X,Y); \rho(rX+s,tY+u) = \begin{cases}-\rho(X,Y),if~rt<0\\ \rho(X,Y),if~rt>0\end{cases}$\\
\hline
\end{tabular}

\paragraph*{Puzzle:} Why does joint distribution exist? (Weight, Height can also be measured jointly.)


\section{Sample Questions}
\paragraph*{10.1} Random variables $X$ and $Y$ have the following joint probabilities: 
\begin{align*}
P(X=0,Y=0)=P(X=0,Y=8)=P(X=16,Y=0)=P(X=16,Y=16)=\frac{1}{4}
\end{align*}
\subparagraph*{a.} Make a table of the joint distribution of $X$ and $Y$ and add their marginal distributions to the table.
\begin{table}[h!]
\centering
\renewcommand{\arraystretch}{1.5}
\begin{tabular}{cccccc} \hline \hline 
%\multirow{2}{*}{a} 
& \multicolumn{3}{c}{b} \\ \cline{2-4}
a  & 0 & 8 & 16 & P(X=a) \\ \hline
0 & $\frac{1}{4}$ & $\frac{1}{4}$ & 0 & $\frac{1}{2}$ \\ 
16 & $\frac{1}{4}$ & 0 & $\frac{1}{4}$ & $\frac{1}{2}$ \\  \hline
P(Y=b) & $\frac{1}{2}$ & $\frac{1}{4}$ & $\frac{1}{4}$ & 1 \\ \hline \hline
\end{tabular}
\end{table}

\subparagraph*{b.} Compute $Cov(X, Y)$. Are $X$ and $Y$ positively correlated, negative corrected, or uncorrelated ?
{\it Solution 1}

{\bf Definition} Let $X$ and $Y$ be two random variables. The {\it covariance} between $X$ and $Y$ is defined by 
\begin{align*}
Cov(X, Y) = E\left[ \left(X - E[X]\right)\left(Y - E[Y]\right)\right]
\end{align*}
First, 
\begin{align*}
E[X] = & 0 \times \frac{1}{2} + 16 \times \frac{1}{2} = 8 \\
E[Y] = & 0 \times \frac{1}{2} + 8 \times \frac{1}{4} + 16 \times \frac{1}{4} = 6
\end{align*}
Let $X^\prime = X - E[X], Y^\prime = Y - E[Y]$, then, we construct a table of the joint distribution of $X^\prime$ and $Y^\prime$ here,
\begin{table}[h!]
\centering
\renewcommand{\arraystretch}{1.5}
\begin{tabular}{cccccc} \hline \hline
& \multicolumn{3}{c}{b} \\ \cline{2-4} 
a & -6=0-6 & 2=8-6 & 10=16-6 & P($X^\prime$=a) \\ \hline
-8=0-8 & $\frac{1}{4}$ & $\frac{1}{4}$ & 0 & $\frac{1}{2}$ \\ 
8=16-8 & $\frac{1}{4}$ & 0 & $\frac{1}{4}$ & $\frac{1}{2}$ \\  \hline
P($Y^\prime$=b) & $\frac{1}{2}$ & $\frac{1}{4}$ & $\frac{1}{4}$ & 1 \\ \hline \hline
\end{tabular}
\end{table}

From the definition, we know that $Cov(X, Y) = E\left[ \left(X - E[X]\right)\left(Y - E[Y]\right)\right]$. By replacing $X - E[X]$ with $X^\prime$ and $Y - E[Y]$ with $Y^\prime$, we obtain $Cov(X, Y) = E[X^\prime Y^\prime]$. 

{\bf TWO-DIMENSIONAL CHANGE-OF-VARIABLE FORMULA} 
Let $X$ and $Y$ be random variables, and let $g: \mathbb{R}^2 \rightarrow \mathbb{R}$ be a function. If $X$ and $Y$ are {\it discrete} random variables with values $a_1, a_2, \ldots$ and $b_1, b_2, \ldots$ respectively, then 
\begin{align*}
E\left[g\left(X, Y\right)\right] = \sum_i \sum_j g(a_i, b_j) P(X = a_i, Y = b_j) .
\end{align*}
If $X$ and $Y$ are {\it continuous} random variables with joint probability density function $f$, then 
\begin{align*}
E\left[g\left(X, Y\right) \right] = \int_{-\infty}^{\infty} \int_{-\infty}^{\infty} g(x, y) f(x, y) dx dy.
\end{align*}

In order to compute the $E[X^\prime Y^\prime]$, we can use the two-dimensional change-of-variable formula by setting $g(X^\prime, Y^\prime) = X^\prime Y^\prime$. Thus, 
\begin{align*}
E\left[X^\prime Y^\prime\right] &= \sum_{a \in \{-8, 8\}} \sum_{b \in \{-6, 2, 10\}} g( a, b) P(X^\prime = a, Y^\prime = b)  \\
& = \sum_{a \in \{-8, 8\}} \sum_{b \in \{-6, 2, 10\}}  abP(X^\prime = a, Y^\prime = b)\\
& =  (-8)(-6)(\frac{1}{4}) + (-8)(2)(\frac{1}{4}) + (8)(-6)(\frac{1}{4}) + (8)(10)(\frac{1}{4}) \\
& = 12 + (-4) + (-12) + (20) \\
& = 16
\end{align*}
Thus, $Cov(X, Y) = 16$. 

{\it Solution 2:}

{\bf AN ALTERNATIVE EXPRESSION FOR THE COVARIANCE.} Let $X$ and $Y$ be two random variables, then 
\begin{align*}
Cov(X, Y) = E[XY] - E[X]E[Y]. 
\end{align*}
From solution 1, we know that $E[X] = 8, E[Y] = 6$, we then compute $E[XY]$. We can compute it by using the two-dimensional change-of-variable formula by setting $g(x, y) = xy$. Then,
\begin{align}
E[XY] & = \sum{a \in \{0, 16\}} \sum_{b \in \{0, 8, 16\}} g(a, b) P(X =a, Y = b) \\
& = (0)(0)(\frac{1}{4}) + (0)(8)(\frac{1}{4}) + (16)(0)(\frac{1}{4}) + (16)(16)(\frac{1}{4}) \\
& = 64 
\end{align}
Thus, 
\begin{align*}
Cov(X, Y) & = E[XY] - E[X]E[Y] \\
& = 64 - (8)(6) \\
& = 16
\end{align*}

If $Cov(X, Y) > 0$, $X$ and $Y$ are positively correlated. \\
If $Cov(X, Y) < 0$, $X$ and $Y$ are negatively correlated. \\
If $Cov(X, Y) = 0$, $X$ and $Y$ are uncorrelated. \\
Since $Cov(X, Y) = 16 > 0$, then $X$ and $Y$ are positively correlated. 

\subparagraph*{c.} Compute the correlation coefficient between $X$ and $Y$. 
{\bf Definition} Let $X$ and $Y$ be two random variable. The {\it correlation coefficient} $\rho(X, Y)$ is defined to be 0 if $Var(X) = 0$ or $Var(Y) = 0$, and otherwise 
\begin{align*}
\rho(X, Y) = \frac{Cov(X, Y)}{\sqrt{Var(X)Var(Y)}}.
\end{align*}

We compute $Var(X) = E[X^2] - \left( E[X]\right) ^2$, while 
\begin{align*}
E[X^2] &  = (0^2)(\frac{1}{2}) + (16^2) (\frac{1}{2}) = 128 \\
E[X] & = 8 \\
Var(X) & =  E[X^2] - \left( E[X]\right) ^2 = 128 - 8^2 = 64 
\end{align*} 
Similarly, we compute $Var(Y) = E[Y^2] - \left( E[Y]\right) ^2$ such that
\begin{align*}
E[Y^2] &  = (0^2)(\frac{1}{2}) + (8^2) (\frac{1}{4}) + (16^2)(\frac{1}{4}) = 0 + 16 + 64 = 80\\
E[Y] & = 6 \\
Var(Y) & =  E[Y^2] - \left( E[Y]\right) ^2 = 80 - 6^2 = 44 
\end{align*} 
We know that $Cov(X, Y) = 16$. Thus, 
\begin{align*}
\rho(X, Y) & = \frac{Cov(X, Y)}{\sqrt{Var(X)Var(Y)}} \\
& = \frac{16}{\sqrt{(64)(44)}} \\
& = \frac{\sqrt{11}}{11} \\
& \approx 0.3015
\end{align*}

\paragraph*{10.5} Suppose $X$ and $Y$ are discrete random variables taking values 0, 1, and 2. The following is given about the joint and marginal distributions:
\begin{table}[h!]
\centering
\renewcommand{\arraystretch}{1.5}
\begin{tabular}{cccccc}
\hline \hline 
 &  \multicolumn{3}{c}{a} \\ \cline{2-4} 
b & 0 & 1 & 2 & P(Y=b) \\ \hline
0 & $\frac{8}{72}$ & $\ldots$ & $\frac{10}{72}$ & $\frac{1}{3}$ \\ 
1 & $\frac{12}{72}$ & $\frac{9}{72}$ & $\ldots$ & $\frac{1}{2}$ \\  
2 & $\ldots$ & $\frac{3}{72}$ & $\ldots$ & $\ldots$ \\ \hline
P(X=a) & $\frac{1}{3}$ & $\ldots$ & $\ldots$ & 1 \\ \hline \hline
\end{tabular}
\end{table}

\subparagraph*{a. Complete the table.}
\begin{align*}
P(X=1, Y= 0) & = P(Y=0) - P(X= 0, Y = 0) - P(X = 2, Y = 0) = \frac{1}{3} - \frac{8}{72} - \frac{10}{72} \\
& = \frac{6}{72} \\
P(X = 0, Y = 2) & = P(X = 0) - P(X = 0, Y = 0) - P(X= 0, Y = 1) = \frac{1}{3} - \frac{8}{72} - \frac{12}{72} \\
& = \frac{4}{72} \\
P(X = 2, Y = 1) & = P(Y = 1) - P(X = 0, Y = 1) - P(X = 1, Y = 1) = \frac{1}{2} - frac{12}{72} - \frac{9}{72} \\
& = \frac{15}{72} \\
P(X = 1) & = P(X = 1, Y = 0) + P(X = 1, Y = 1) + P(X = 1, Y = 2) = \frac{6}{72} + \frac{9}{72} + \frac{3}{72} \\
& = \frac{1}{4} \\
P(Y = 2) &  = 1 - P(Y = 0) - P(Y = 1) = 1 - \frac{1}{3} - \frac{1}{2} \\
& = \frac{1}{6} \\
P(X = 2, Y = 2) & = P(Y = 2) - P(X = 0, Y = 2) - P(X = 1, Y = 2)  = \frac{1}{6} - \frac{4}{72} - \frac{3}{72} \\
& = \frac{5}{72} \\
P(X = 2) & = P(X = 2, Y = 0) + P(X = 2, Y = 1) + P(X = 2, Y = 2) = \frac{10}{72} + \frac{15}{72} + \frac{5}{72} \\
& = \frac{5}{12}
\end{align*}
Thus, the completed table is 
\begin{table}[h!]
\centering
\renewcommand{\arraystretch}{1.5}
\begin{tabular}{cccccc}
\hline \hline 
 &  \multicolumn{3}{c}{a} \\ \cline{2-4} 
b & 0 & 1 & 2 & P(Y=b) \\ \hline
0 & $\frac{8}{72}$ & $\frac{6}{72}$ & $\frac{10}{72}$ & $\frac{1}{3}$ \\ 
1 & $\frac{12}{72}$ & $\frac{9}{72}$ & $\frac{15}{72}$ & $\frac{1}{2}$ \\  
2 & $\frac{4}{72}$ & $\frac{3}{72}$ & $\frac{5}{72}$ & $\frac{1}{6}$ \\ \hline
P(X=a) & $\frac{1}{3}$ & $\frac{1}{4}$ &  $\frac{5}{12}$ & 1 \\ \hline \hline
\end{tabular}
\end{table}

\subparagraph*{b.} Compute the expectation of $X$ and of $Y$ and the covariance between $X$ and $Y$.  
\begin{align*}
E[X]  & = \sum_{a \in \{0, 1, 2\}} aP(X = a) \\
& = (0)(\frac{1}{3}) + (1)(\frac{1}{4}) + (2)(\frac{5}{12}) \\
& = \frac{13}{12} \\
E[Y] & = \sum_{b \in \{0, 1, 2\}} b P(Y=b) \\
& = (0)(\frac{1}{3}) + (1) (\frac{1}{2}) + (2)(\frac{1}{6}) \\
& = \frac{5}{6}
\end{align*}

In order to compute the $Cov(X, Y)$, we can use $Cov(X, Y) = E[XY] - E[X]E[Y]$. We then have to compute $E[XY]$. We can use the two-dimensional change-of-variable formula by setting $g(x, y) = xy$. Thus, 
\begin{align*}
E[XY] & = \sum_{a\in \{0, 1, 2\}} \sum_{b\in \{0, 1, 2\}} g(a, b) P(X = a, Y = b) \\
& = (1)(1)(\frac{9}{72}) + (1)(2)(\frac{15}{72}) + (2)(1)(\frac{3}{72}) + (2)(2)(\frac{5}{72}) \\
& = \frac{65}{72}
\end{align*}
Thus, 
\begin{align*}
Cov(X, Y) & = E[XY] - E[X]E[Y] = \frac{65}{72} - \frac{13}{12}\frac{5}{6} \\
& = 0
\end{align*}

\subparagraph*{c.} Are $X$ and $Y$ are independent ? 

{\bf Question Here ?} Can we say that $X$ and $Y$ are independent because $X$ and $Y$ are uncorrelated ? 
{\bf NO!!!}

{\bf INDEPENDENCE VERSUS UNCORRELATED.} If two random variable $X$ and $Y$ are independent, then $X$ and $Y$ are uncorrelated. \\
{\bf Note that the reverse is not necessarily true.} If $X$ and $Y$ are uncorrelated, they need not be independent. 

Remember the definition: 
{\bf DEFINITION} An event $A$ is called {\it independent} of $B$ if $P(A|B) = P(A)$. \\
{\bf INDEPENDENCE.} To show that $A$ and $B$ are independent it suffice to prove {\it just one} of the following: 
\begin{align*}
P(A|B) & = P(A) \\
P(B|A) & = P(B) \\
P(A \cap B ) & = P(A) P(B), 
\end{align*}
where $A$ may be replaced by $A^c$ and $B$ replaced by $B^c$, or both. IF one of these statements holds, {\it all} of them are true. If two events are not independent, they are called {\it dependent}. 

{\bf To show Independent}If you want to show that $A$ and $B$ are independent, you can justified it by mathematically showing that one of those statements holds. Or, you can check one of those statements. It holds for {\it all} possible outcomes of $A$ and {\it all} possible outcomes of $B$. Then you can say that $A$ and $B$ are independent. 

{\bf To show Dependent} You can just show for one outcome of $A$ (e.g., a) and one outcome of $B$ (e.g., b) that one of those statements doesn't hold. {\bf Anyone of the pairs with specific value of a and specific value of b}. 

For this question, these two variables are actually independent. 
Yes, $X$ and $Y$ are independent. We can check for any $a$ and $b$, $P(X = a, Y = b) = P(X = a) P(Y = b)$. 
\end{document}
