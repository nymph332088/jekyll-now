% TEMPLATE FOR STANDARD QUIZ
% by laura

\documentclass[11pt,epsfig]{article}
\usepackage{amsmath} 
\usepackage{mdframed}
\usepackage{graphicx}
\usepackage{multirow}
\def\infinity{\rotatebox{90}{8}}
\oddsidemargin=0in
\evensidemargin=0in
\textwidth=6.3in
\topmargin=-0.5in
\textheight=9in
\renewcommand{\arraystretch}{2}
\parindent=0in
\pagestyle{empty}

\input{notestpoints}

\begin{document}


%%%(change to appropriate class and semester)
\textbf{Notes for Chapter 7: CIS 2033 Fall 2015 Computational Prob and Stat}

%%%(change to appropriate quiz type and date)
\textbf{Shanshan Zhang {(tuf14438@temple.edu)} }


% problem
\section{Chapter 7: Expectation and Variance}
\subsection{Definition}
\begin{tabular}{|l|}
\hline
\textbf{Definition:}\\
$E[X]= \begin{cases} \sum_xxP(x), & X~ \text{is discrete}\\
\int_x xf(x)dx, &X~ \text{is continuous} \end{cases},Var[X]= E[(X-E[X])^2]$\\
\hline
\end{tabular}

\subsubsection{Important expectation and variance}
\begin{tabular}{ |l|l|l|l| }
%\caption{Important Expectation and Variance}
\hline
& \bf{Notation }& \bf{$E[X]$} & \bf{$Var[X]$}\\ \hline
\multirow{3}{*}{Discrete} & $X\sim Ber(p)$ & $p$ & $p(1-p)$\\ \cline{2-4}
 & $X\sim Bin(n,p)$ & $np$& $np(1-p)$\\  \cline{2-4}
 & $X\sim Geo(p)$ & $\frac{1}{p}$& $\frac{1-p}{p^2}$\\ \cline{2-4}
 & $X\sim Pois(\mu)$ & $\mu$ & $\mu$\\ \hline
\multirow{4}{*}{Continuous} & $X\sim Unif(a,b)$ & $\frac{a+b}{2}$& $\frac{(b-a)^2}{12}$\\ \cline{2-4}
 & $X\sim Exp(\lambda)$ & $\lambda^{-1}$& $\lambda^{-2}$\\ \cline{2-4}
 & $X\sim Par(\alpha)$ & $\frac{\alpha}{\alpha-1}$& $\frac{\alpha}{(\alpha-1)^2(\alpha-2)},for ~\alpha >2$\\ \cline{2-4}
& $X\sim N(\mu,\sigma^2)$&$\mu$&$\sigma^2$\\ \hline
\end{tabular}

\subsection{Change of variables}
Exercise 7.3, 7.4 \\
\begin{tabular}{|l|}
\hline
\textbf{Definition:} $E[g(X)] =\begin{cases} \sum_xg(x)P(X=x), & X~ \text{is discrete}\\ \int_x g(x)f(x)dx, &X~ \text{is continuous} \end{cases};Var[g(X)]= E[(g(X)-E[g(X)])^2]$\\
\textbf{Useful formulas:}: $Var[X]=E[X^2]-(E[X])^2;E[aX+b] = aE[X]+b; Var[aX+b] = a^2Var[X] $\\
\hline
\end{tabular}
\end{document}