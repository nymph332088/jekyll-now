\documentclass{article} % For LaTeX2e
\usepackage{nips12submit_e,times}
\usepackage{graphicx}
\usepackage{amsmath}
\usepackage{amssymb}
%\documentstyle[nips12submit_09,times,art10]{article} % For LaTeX 2.09


\title{Homework \\
Chapter 6}


\author{
Min Xiao\\
Department of Computer and Information Sciences\\
Temple University\\
Philadelphia, PA 19122 \\
\texttt{minxiao@temple.edu} \\
}

% The \author macro works with any number of authors. There are two commands
% used to separate the names and addresses of multiple authors: \And and \AND.
%
% Using \And between authors leaves it to \LaTeX{} to determine where to break
% the lines. Using \AND forces a linebreak at that point. So, if \LaTeX{}
% puts 3 of 4 authors names on the first line, and the last on the second
% line, try using \AND instead of \And before the third author name.

\newcommand{\fix}{\marginpar{FIX}}
\newcommand{\new}{\marginpar{NEW}}

\nipsfinalcopy % Uncomment for camera-ready version

\begin{document}

\maketitle

\paragraph*{6.1} Let $U$ have a $U(0, 1)$ distribution. 

\subparagraph*{a.} Describe how to simulate the outcome of a roll with a die using $U$. 

\begin{table}[h!]
\renewcommand{\arraystretch}{1.5}
\centering
\caption{Simulation of rolling a die using $U(0, 1)$.}
\label{Ta:6.1a}
\begin{tabular}{|c|c|c|c|c|c|c|} \hline
U & $0\leq U < \frac{1}{6}$ & $\frac{1}{6} \leq U < \frac{2}{6}$ & 
$\frac{2}{6} \leq U < \frac{3}{6}$ & $\frac{3}{6} \leq U < \frac{4}{6}$ & 
$\frac{4}{6} \leq U < \frac{5}{6}$ & $\frac{5}{6} \leq U \leq 1$ \\ \hline 
N & 1 & 2 & 3 & 4 & 5 & 6 \\ \hline 
\end{tabular}
\end{table}

\subparagraph*{b.} Define $Y$ as follows: round $6U+1$ down to the nearest integer. What are the possible outcomes of $Y$ and their probabilities ? 

Since $0 \leq U \leq 1$, then $1 \leq 6U + 1 \leq 7$. 
\begin{table}[h!]
\renewcommand{\arraystretch}{1.5}
\centering
\caption{Y.}
\label{Ta:6.1b}
\begin{tabular}{|c|c|c|c|c|c|c|c|} \hline
$U$ & $0\leq U < \frac{1}{6}$ & $\frac{1}{6} \leq U < \frac{2}{6}$ & 
$\frac{2}{6} \leq U < \frac{3}{6}$ & $\frac{3}{6} \leq U < \frac{4}{6}$ & 
$\frac{4}{6} \leq U < \frac{5}{6}$ & $\frac{5}{6} \leq U < 1$ & $U=1$\\ \hline 
$Z=6U + 1$ & $1\leq Z < 2$ & $2 \leq Z <3$ & 
$3 \leq Z < 4$ & $4 \leq Z < 5$ & 
$5 \leq Z < 6$ & $6 \leq Z < 7$ & 7\\ \hline 
$Y=\lfloor Z \rfloor$ & 1 & 2 & 3 & 4 & 5 & 6 & 7 \\ \hline
P(Y) & $\frac{1}{6}$ & $\frac{1}{6}$ & $\frac{1}{6}$ & $\frac{1}{6}$ & $\frac{1}{6}$ & $\frac{1}{6}$ & 0 
\\ \hline 
\end{tabular}
\end{table}

\paragraph*{6.3} Let $U$ have a $U(0, 1)$ distribution. Show that $Z = 1 - U$ has a $U(0, 1)$ distribution by deriving the probability density function or the distribution function. 

Since $0 \leq U \leq 1$, then $0 \leq Z \leq 1$. 
Let $0 \leq a \leq 1$, 
\begin{align*}
F_Z(a) & = P(Z \leq a) \\
&  = P(1-U \leq a)\\
&   = P(U \geq 1 - a) \\
& = 1 - P(U \leq 1 - a) \\
&  = 1 - F_U(1 - a) \\
& = 1 - (1 - a) \\
& = a \\
f(z) & = \frac{d F(Z)}{dZ} \\
& = 1 
\end{align*}
Now, we know that $Z$ is a continuous random variable between zero and one, $0 \leq Z \leq 1$, and its density function is $f(z) = 1$ for $0 \leq z \leq 1$, and $f(z) = 0$ for elsewhere. Thus, $Z \sim U(0, 1)$. 
 
\paragraph*{6.6} Somebody messed up the random number generator in your computer, instead of uniform random numbers it generates numbers with an $Exp(2)$ distribution. Describe how to construct a $U(0,1)$ random variable $U$ from an $Exp(2)$ distributed $X$. 

For the uniform distribution, $U(0, 1)$, the distribution function is $F_U(u) = u$ for $0 \leq u \leq 1$. For the exponential distribution, $Exp(2)$, the distribution function is $F_X(x) = 1 - e^{-2x}$ for $x \geq 0$. In order to construct a $U(0, 1)$ random variable $U$ from an $Exp(2)$ distributed $X$, we have to compute $F_U(u) = F_X(x)$. This means $u = 1-e^{-2x}$ for $x \geq 0$. Then, we can define the $U(0, 1)$ distributed random variable $U = F^{inv}(X) = 1-e^{-2X}$. If $U$ is a uniform distribution, then $1-U$ is also a uniform distribution, we can also construct $U = e^{-2X}$. 

\paragraph*{6.8} A random variable X has a $Par(3)$ distribution, so with distribution function $F$ with $F(x) = 0$ for $x < 1$, and $F(x) = 1-x^{-3}$ for $x \geq 1$. For details of the Pareto distribution see Section 5.4. Describe how to construct $X$ from $U(0, 1)$ random variable. 

For the $U(0, 1)$ random variable $U$, the distribution function is $F_U(u) = u$ for $0 \leq u \leq 1$. For the $Par(3)$ random variable $X$, the distribution function is $F_X(x) = 1 - \frac{1}{x^{3}}$, for $x \geq 1$. In order to construct a $Par(3)$ random variable $X$ from a $U(0, 1)$ distributed variable $U$, we have to compute $F_X(x) = F_U(u)$. This means $1 - \frac{1}{x^3} = u$. Then we can get $x = (1-u)^{-\frac{1}{3}}$, for $0 \leq u \leq 1$. Then, we can define the $Par(3)$ distributed random variable $X = (1 - U)^{-\frac{1}{3}}$. Since $Y = 1-U$ is also a $U(0, 1)$ distributed random variable is $U$ is a $U(0, 1)$ distributed random variable, we can also construct $X = U^{-\frac{1}{3}}$.  

\paragraph*{6.9} In Quick Exercise 6.1 we simulated a die by tossing three coins. Recall that we might need several attempts before succeeding. 

If tossing three coins, we could get eight possible outcomes, {HHH, HHT, HTH, THH, HTT, THT, TTH, TTT}. For a die, there should be six outcomes, {1, 2, 3, 4, 5, 6}. Thus, we could match

\begin{table}[h!]
\renewcommand{\arraystretch}{1.5}
\centering
\caption{Simulation of rolling a die using $U(0, 1)$.}
\label{Ta:6.1a}
\begin{tabular}{|c|c|c|c|c|c|c|} \hline
three coins & HHH & HHT & HTH & THH & HTT & THT \\ 
die & 1 & 2 & 3 & 4 & 5 & 6 \\ \hline
\end{tabular}
\end{table}
If the outcome is HHT, or TTT, we toss those three coins again until we meet the other six outcomes. 

\subparagraph*{a.} What is the probability that we succeed on the first try ?

The probability is $\frac{6}{8} = \frac{3}{4}$. 

\subparagraph*{b.} Let $N$ be the number of tries that we need. Determine the distribution of $N$. 

We know each trial is a $Ber(\frac{3}{4})$. We repeated it until succeed. We know $N \sim Geo(\frac{3}{4})$. 

\end{document}
